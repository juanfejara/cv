\section{Experiencia Laboral}

\begin{entrylist}
  \entry
    {06/24-Now}
    {Ingeniero de Software, Remoto, Senior}
    {EPAM}
    {Clean code de la aplicación MVA de flujo de tareas, Python, multiples bases de datos. Ingeniero en departamento de investigación
	desarrollando RAGs}
  \entry
    {08/23-04/24}
    {Ingeniero de Datos, Remoto, Senior}
    {Sonder {\sl Industria Hotelera}}
        {Sonder es una compañía de la industria del alojamiento. Hice un plugin \href{http:\\https://github.com/juanfejara/sqlfluff-templ-check}{plugin} para la aplicación SQLFluff que forzaba el uso de ref en SQL.
	Migración de código legacy de ETLs en DBT a Snowflake.}
  \entry
    {05/21-05/23}
    {Ingeniero de Datos, desarrollador backend, Remoto, Senior}
    {Ebury {\sl Fintech de cambio de monedas}}
    {Ebury es una Fintech con base en Londres y Málaga, perteneciente al grupo Santander, que realiza intercambio de divisas con presencia en 24 países.

Mi trabajo consistió en la migración del ORM del backoffice en Django a BigQuery usando dbt y Apacha Airflow. Optimización de reportes, y ETLs usando Python, Bash, o SQL según la necesidad, ya sea creando contenedores Dockers, que corrían en Kubernetes, o procedimientos en Confluence y Kafka. La infraestructura está basada parcialmente en AWS y parte en GCloud.}
	\entry

    {03/20-12/22}
    {Desarrollador Python, Científico de Datos, Senior}
    {Globant {\sl Compañía de desarrollo de Software}}
	{SmileBuilder y SmileTreatment; Software de tratamiento de ortodoncia para Smile Direct Club. Hecho en C++ y Java,
fuerte desarrollo usando Geometría Diferencial para reproducir mandíbulas en 3D.

Complementos desarrollados para DataikuDSS mejorando la funcionalidad y haciendo
Código limpio en el equipo de Data Science de Johnson \& Jhonson. Complementos en Python y Clean Code.}
  \entry
    {01/16-11/19}
    {Científico de Datos, Programador Fullstack}
    {OpDevel {\sl Compañía de consultoría y desarrollo}}
    {Prevención de Fraudes \textbf{Tpaga}, reducción del fraude en un 7\% en la Walled App, mediante la
creación de una nueva metodología innovadora.

Procesamiento de datos, desarrollador Python. \textbf{Crossover.com} Compañía de trabajo remoto

Consultor desarrollador. Certicamara, empresa certificadora. Java8, bajo
DropWizard como backend y NodeJS como frontend Arquitectura REST de NodeJS,
usando Mongo, RabbitMQ y ayudando en el uso de AWS y GCP por Kubernetes
y Estibadores.

Científico de datos. Aplicación Cívico. Análisis de datos, introduje el análisis de datos usando
software de física innovador como ROOT y Jupyter, produciendo un gran impacto en
la empresa.

Gerente de proyecto. Sistema de seguimiento y transmisión de video para vehículos.
Consultor. Sistema de captura de video y reconstrucción de escena 3D del vehículo
accidentes

%Diseño conceptual del presupuesto de infraestructura de TI en la concesión de peaje Honda - Girardot.

Software de transporte Caracol en Haití. Ingenico desarrollador del sistema de tarjetas Transmilenio,
consultor de Recaudos Bogotá, para la empresa GSD+.}
  \entry
	{02/13-12/15}
	{P. Owner, P. Manager, Programador Senior}
	{Brückner {\sl Compañía constructora de plantas laminadoras de plástico Alemana}}
	{Product Owner de PEM, software MRP. Desarrollar un modelo para mejorar el proceso de los clientes.

	Desarrollo de un MRP, basado en Odoo para las plantas de laminado de plástico.}

\end{entrylist}
\newpage
%\begin{aside}
    ~
    ~
  \section{Habilidades Personales}
    \smartdiagram[bubble diagram]{
	\textbf{Enfoco en}\\\textbf{la gente}\\\textbf{Trabajo en}\\\textbf{equipo},
        \textbf{Aprendizaje}\\\textbf{rápído},
        \textbf{Motivador},
	\textbf{Iniciador},
        \textbf{Gestor de}\\\textbf{bloqueos},
        \textbf{Collaborador},
        \textbf{Resolución de}\\\textbf{Problemas}
    }
    ~
    ~
  \section{Product Owning}
    \smartdiagram[bubble diagram]{
      \textbf{Communication}\\\textbf{Skills},
      \textbf{Data}\\\textbf{Visualization},
      \textbf{Decision}\\\textbf{Making},
      \textbf{Open}\\\textbf{Communication},
      \textbf{Initiative},
      \textbf{Active}\\\textbf{Listening},
      \textbf{Problem}\\\textbf{Understanding},
      \textbf{Reading}\\\textbf{Skills},
      \textbf{Humility},
      \textbf{Fast}\\\textbf{Learner\vspace{3mm}}
    }
\end{aside}


\begin{entrylist}


  \entry
	{06/10-02/13}
	{Científico de Datos, Research Programmer}
	{EmQbit {\sl Compañía orientada a la Technology}}
	{Científico de Datos para Detección de Riesgo y Fraude en el proyecto DetectTA, y DTA.
Desarrollar aplicación Web, bajo Google App Engine, para la gestión de hipotecas.

Desarrollo de modelos matemáticos para asignaciones hipotecarias.

Desarrollo de aplicaciones para la seguridad de las máquinas de votación que utilizan criptografía
en el microvKernel de Linux en el momento del arranque.

App para el Auto préstamo de Libros, en la biblioteca de la Universidad Nacional.}

  \entry
	{01/07-06/10}
	{CTO, Programador}
	{Mig Internacional {\sl Compañía orientada a la Technology}}
	{Sistemas de identificación automática con identificación por radiofrecuencia (RFID)
y reconocimiento óptico de caracteres (OCR).

Cámara integrada, usando FPGA para hacer Reconocimiento de placa de línea.

Desarrollo de equipos RFID en Alemania, que utilizan FPAA para procesar la
Señal de Tags.}

  \entry
	{06/04-12/06}
	{Actuario de Casualty}
	{Allianz Colseguros S.A. {\sl Compañía Aseguradora.}}
	{Diseño BD CEXPER, central de riesgo de siniestros.

Análisis de Nota Técnica para Vicepresidencia de Automóviles.

Organización del Segundo Simposio Nacional de Actuaría.

Desarrollador el conector entre el software Tricast a la compañía.

Evaluación de software para el equipo actuarial, Emblem y Tricast.}


  \entry
	{12/02-01/04}
	{Statistical QC}
	{PGS Onshore {\sl Compañía Sísmica}}
	{Análisis de estadísticas. Desarrollo de ERP (Pachallaqta) para gestionar proyectos
	usando sísmica 3D en C++ y Oracle.}

  \entry
	{01/96-12/01}
	{IT Manager, Desarrollador}
	{Xania LTDA. {\sl Compañía de desarrollo de Software}}
	{Desarrollo de Ports, software para el cargue descargue en buques y medición de pilas de carbón,
	para la compañía Intercor, ahora Cerrejón S.A.

Software para valoración de siniestros en vehículos a motor (Palm Pilot), Empresa
Subocol.

Software Infoban: para el Ministerio de Justicia (Web - Oracle Web Server, C,
C++)

%Soporte ERP y migración del núcleo ERP de Pascal a Delphy,
%Gerente de Desarrollo de Edición de Software y Desarrollo de Edificios Educativos.
%Software educativo. Empresa Telemundo Editores. Uso de Macromedia
%productos “Authorware, Director”, Visual Basic, Pascal y Borland Delphy
%C.

%Creación del emulador para NDC+ (protocolo de NCR y Diebold ATM) para
%ATM Tryton usando C++. Empresa RK Technologies, empresa dedicada a
%la producción de cajeros automáticos y software bancario.
%Construí y programé la automatización del robot para el Museo del Niño en Bogotá.
%Empresa Microservicios.

%Desarrollo de un algoritmo de búsqueda para software DOS y desarrollo de hipertexto
%software de texto para Windows 3.1 usando Borland C++. Empresa Leginfo Ltda.
}

  \end{entrylist}
