\section{Open Source and Collaborative Projects}

The software presented here is covered by GPL or is given to public
domain and can be donwloadable for free over the Internet, I also have
made some translations to spanish of different projects to ease
the use of technology in Colombia and other countries.

\begin{entrylist}
  \entry
    {2017-}
    {Canair.io}
    {Air Quality Measure citizen science}
    {Currently aiming to build a network of people helping
    to compare the official reports with the data produced by devices and citizen science}
  \entry
    {2011-}
    {Openstreetmap Colombia}
    {Making Openstreetmap usable in Colombia}
    {We have developed some tools and have made contributions
    with data and support people that want to have a free tools.}
  \entry
    {2000-2004}
    {Cuenta}
    {Accounting Software for service companies}
    {Maintainer for this double entry account
    software that is subject to the colombian norms of accounting, and
    that shares the database with gestiona. Part of Structio  project
    http://structio.sf.net . It's on devel state. Implemented on
    PHP and postgresql.}
 \entry
    {2002}
    {Gestiona}
    {Administrative tasks in schools}
    {Software aimed to act as a white pages for school
    including students, workers and suppliers.  Part of the 
    Structio project http://structio.sf.net . Actually it's being
    developed, implemented in python and postgresql.}
  \entry
    {2000}
    {Gtypist}
    {Software to improve typing}
    {maintained for a short period, we also added some lessons suited for schools}
  \entry
    {1999}
    {GFPP}
    {Flower Production Predictor}
    {Created and maintained a
    package used to predict the production of flowers for a
    farm. Developed in Java and tested on Linux and Windows. http://gfpp.sf.net}
\end{entrylist}