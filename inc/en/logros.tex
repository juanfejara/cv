\section{Publications}
\begin{entrylist}
  \entry
    {2000-2003}
    {\href{http://structio.sourceforge.net/guias/AA_Linux_colegio/AA_Linux_colegio.html}{Guías Para aprender a Aprender GNU/Linux}}
    {Proyecto Structio}
    {Learn to use GNU/Linux}
  \entry
    {2000}
    {Mathematical and numerical study of an
    elasticity problem}
    {Universidad de los Andes}
    {Undergraduate Thesis for Maths}
  \entry
    {2000}
    {PLeNa: An approach to Natural Language
    Processing}
    {Universidad de los Andes}
    {Thesis for Systems Engineering}
\end{entrylist}

\section{Presentations, Speeches and Talks Given}
\begin{entrylist}
  \entry
    {10/2018}
    {Pyday Cali}
    {Using Python to integrate a Digital Wallet}
    {This talk showed how Tpaga could be integrated in an ecommerce}
  \entry
    {08/2018}
    {SLUD}
    {Professional Paths for System Egineers}
    {For undergraduate students possible paths for professional growth}
  \entry
    {02/2017}
    {Segundo Festival de Cultura Libre}
    {Living with Free Software}
    {Entrepeneurs and students can make use of Free Software to turn on Startups}
  \entry
    {10/2015}
    {Bogotá Ruby Meetup}
    {Use case: MVPs, scaling and balancing in a startup}
    {Ruby on Rails is heavily used in Tappsi, we show the tools and the current load}
\end{entrylist}
\begin{entrylist}
  \entry
    {12/2014}
    {Django Bogotá Meetup}
    {Django, Reports and printing}
    {Tips on how to get from the web to printing.}
  \entry
    {10/2012}
    {Python Colombia Community}
    {A glimpse of Flask}
    {Presenting an integration  of Openstreetmap with Flask and
    comparison with other solutions}
  \entry
    {06/2011}
    {Campus Party Colombia}
    {Mercurial - The easy way}
    {How you can have your own distributed version control in an easy way}
  \entry
    {10/2008}
    {Congreso Internacional de Software Libre y
    democratización del conocimiento}
    {Colombian Case Studies}
    {Presentation of some of the uses of free software in Colombian
    Education Institutions}
  \entry
    {06/2006}
    {Feria de educacion y tecnologia libre}
    {Free contents production}
    {Some experiences learned about the production of material
    by school students.}
  \entry
    {10/2005}
    {Local Forum}
    {Students producing contents}
    {Show of
    some products of programation and methodology to accomplish such task}
  \entry
    {05/2005}
    {UTP}
    {SLEC presentation and  Linux at GFC}
    {The SLEC
    community was presented and also showed a successful implementation of
    Linux at a school}
  \entry
    {10/2004}
    {SLUD 3}
    {Python in Education}
    {Presented Python as an
    appropiate language to articulate a curriculum in schools and many
    courses at University}
  \entry
    {04/2003}
    {SLEC Conferences}
    {GFC and Free Software}
    {It was shown the GFC as
    study case showing a school 100\% free software}
  \entry
    {09/2001}
    {Speeches at FUSM}
    {Free Software in
    universities}
    {It was showed to students the benefits of using free
    software in their day to day}
  \entry
    {06/2000}
    {Free Software Talks}
    {Free software at schools}
    {It was shown Linux as an operating system usable for schools, GFC as an example}
  \entry
    {05/2000}
    {Foro Distrital de Educación}
    {Free Software in
    schools}
    {It was shown Linux as an operating system usable for schools, GFC as an example}
\end{entrylist}